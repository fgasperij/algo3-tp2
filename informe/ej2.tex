\section{Problema 2: Caballos salvajes}

\subsection{Presentación del problema}
En este problema nos presentan un tablero de dimensiones conocidas pero no acotadas, $n \times n$,
del mismo tipo que el de ajedrez, es decir un tablero cuadriculado:\\
\begin{center}
(TODO insertar imagen de tablero cuadriculado y a los lados llaves que digan 'n')
\end{center}
y un conjunto de caballos, piezas de ajedrez, en casillas también conocidas, por ejemplo:\\
\begin{center}
(TODO insertar imagen del tablero con algunas casillas con una C)
\end{center}
el objetivo es reunir a todos los caballos en una misma casilla del tablero con una cantidad
de movimientos total, es decir sumando los movimientos que le tomo a todos los caballos, mínima.
Los movimientos permitidos a los caballos son los mismos que los impuestos por las reglas de ajedrez:\\
\begin{center}
(TODO insertar imagen del tablero con una C y x's en los casilleros donde puede saltar el caballo)
\end{center}
La C representa la casilla donde está ubicado el caballo y X los lugares a los cuales podría saltar.

\subsection{Resolución}
Nosotros modelamos el problema con grafos con lo cual el problema es equivalente al de camino mínimo 
con un origen y múltiples destinos. En este caso en particular como el salto de un caballo siempre 
tiene el mismo costo basta recorrer el grafo con Breadth First Search (TODO agregar cita) y el camino mínimo queda 
determinado como el camino en el árbol generador resultante.
El modelo consiste en tomar a las casillas del tablero como nodos del grafo, las casillas en las que 
hay caballos como nodos origen y existe una arista entre dos nodos, casillas, si de una a la otra se
puede ir con un salto de caballo. Cabe destacar que el salto de caballo es simétrico: si con un salto
de caballo se puede ir de la casilla $a$ a la $b$ entonces necesariamente también se puede con un 
salto de caballo ir de la casilla $b$ a la $a$:\\
\begin{center}
(TODO insertar imagen de un tablero con la C y las x's y al lado una imagen de un tablero con un puntito en
la C y aristas que lo unen a puntitos donde estaban las x's)
\end{center}
Los pasos para resolver el problema una vez modelado son bastante directos:
\begin{enumerate}
  \item Recorrer el grafo una vez por cada caballo presente tomando como origen el nodo, casilla, en la 
    que se encuentra el caballo.
  \item Sumar, en cada nodo, la cantidad de saltos que le toma a cada caballo llegar a él.
  \item Recorrer todos los nodos y quedarse con alguno de los de suma mínima.
\end{enumerate}

\subsection{Pseudocódigo}
\begin{algorithm}[H]
  \begin{algorithmic}
    %\STATE $\gets$ \WHILE{} \ENDWHILE \IF{} \ELSE \ENDIF
    \STATE $distancias$ $\gets$ Matriz($n$, $n$)
    \STATE $tablero$ $\gets$ Tablero($n$, $n$)
    \STATE $caballos$ $\gets$ Conj($Casilla$)
    \WHILE {$caballos \neq \emptyset$}
      \STATE $origen$ $\gets$ $caballos$.next()
      \STATE BFS($origen$, $tablero$, $distancias$)
    \ENDWHILE
    \STATE ($suma_{min}$, $casilla_{min}$) $\gets$ MIN($distancias$)
    \caption{caballos\_salvajes}
  \end{algorithmic}
\end{algorithm}


\subsection{Demostración de correctitud}



\subsection{Análisis de complejidad}
No es necesario construir el grafo explícitamente ya que los saltos de caballo desde una casilla pueden contruirse 
ad hoc, por lo tanto, dada un nodo, casilla, obtener los vecinos del mismo tiene costo $O(1)$ porque a lo sumo
puede contar con 8 vecinos:
(TODO insertar imagen de un tablero con una C en el centro, otra en un rincón y otra en una banda, indicando en cada
caso los saltos posibles)

\subsection{Tests de complejidad}
