\section{Problema 2: Caballos salvajes}

\subsection{Presentación del problema}
En este problema nos presentan un tablero de dimensiones conocidas pero no acotadas
del mismo tipo que el de ajedrez, es decir un tablero cuadriculado y cuadrado:\\
\begin{center}
(TODO insertar imagen de tablero cuadriculado y a los lados llaves que digan 'n')
\end{center}
y un conjunto de caballos, piezas de ajedrez, en casillas también conocidas, por ejemplo:\\
\begin{center}
(TODO insertar imagen del tablero con algunas casillas con una C)
\end{center}
el objetivo es reunir a todos los caballos en una misma casilla del tablero con una cantidad
de movimientos total, es decir sumando los movimientos que le tomo a cada uno de los caballos, mínima.
Los movimientos permitidos a los caballos son los mismos que los impuestos por las reglas de ajedrez:\\
\begin{center}
(TODO insertar imagen del tablero con una C y x's en los casilleros donde puede saltar el caballo)
\end{center}
(TODO el renglón que está debajo de este debería ser el caption de la anterior imagen.)
La C representa la casilla donde está ubicado el caballo y X los lugares a los cuales podría saltar.

\subsection{Resolución}
Nosotros modelamos el problema con grafos con lo cual resultó equivalente al de camino mínimo 
con un origen y múltiples destinos en un grafo no dirigido sin peso en las aristas. 
En este caso en particular las aristas no tienen peso porque el salto de un caballo siempre 
tiene el mismo costo. En un grafo con las características mencionadas el algoritmo de 
Breadth First Search (TODO agregar cita) obtiene el camino mínimo entre el origen tomado y el resto de los nodos,
éste es el camino entre el origen y el nodo en árbol generado.
El modelo consiste en tomar a las casillas del tablero como nodos del grafo, las casillas en las que 
hay caballos como nodos origen y existe una arista entre dos nodos, casillas, si de una a la otra se
puede ir con un salto de caballo. Cabe destacar que el salto de caballo es simétrico: si con un salto
de caballo se puede ir de la casilla $a$ a la $b$ entonces necesariamente también se puede con un 
salto de caballo ir de la casilla $b$ a la $a$:\\
\begin{center}
(TODO insertar imagen de un tablero con la C y las x's y al lado una imagen de un tablero con un puntito en
la C y aristas que lo unen a puntitos donde estaban las x's)
\end{center}
Los pasos para resolver el problema una vez modelado son bastante directos:
\begin{enumerate}
  \item Recorrer el grafo una vez por cada caballo presente tomando como origen el nodo, casilla, en la 
    que se encuentra el caballo.
  \item Guardar, en cada nodo, la cantidad de saltos que le toma a cada caballo llegar a él.
  \item Recorrer todos los nodos y quedarse con alguno de los de suma mínima.
\end{enumerate}

\subsection{Pseudocódigo}
\begin{algorithm}[H]
  \begin{algorithmic}
    %\STATE $\gets$ \WHILE{} \ENDWHILE \IF{} \ELSE \ENDIF
    \STATE $distancias$ $\gets$ Matriz($n$, $n$)
    \STATE $tablero$ $\gets$ Tablero($n$, $n$)
    \STATE $caballos$ $\gets$ Conj($Casilla$)
    \WHILE {$caballos \neq \emptyset$}
      \STATE $origen$ $\gets$ $caballos$.next()
      \STATE BFS($origen$, $tablero$, $distancias$)
    \ENDWHILE
    \STATE ($suma_{min}$, $casilla_{min}$) $\gets$ MIN($distancias$)
    \caption{caballos\_salvajes}
  \end{algorithmic}
\end{algorithm}


\subsection{Demostración de correctitud}
Una instancia del problema está dada por la dimensión del tablero $n$ y la posición
de los $k$ caballos. Luego de modelarlo tenemos un grafo $G = (V, E)$ tal que $\left\vert{V}\right\vert = n^2$, uno
por cada nodo, y $\left\vert{E}\right\vert < n^2 * 4$. La cota superior sobre las aristas proviene del hecho de que
un caballo sólo pude saltar a 8 casillas diferentes como máximo, asumiendo que todos los
saltos posibles caen dentro del tablero. Por lo tanto, la cantidad de aristas debe ser menor
a $\frac{n^2 * 8}{2}$ porque en muchas casillas, como las de las esquinas o las de las bandas, la
mayoría de los saltos posibles caen fuera del tablero. Los caballos los representaremos con un multiconjunto
de nodos al que llamaremos $origenes$, $\left\vert{origenes}\right\vert = \left\vert{caballos}\right\vert = k$. Utilizamos un multiconjunto 
para representarlos porque inicialmente puede haber más de un caballo en una casilla determinada.
Sea $PS_{v,u}$ el conjunto de los caminos posibles entre el nodo $u$ y el nodo $v$.
Dado un nodo $v$ el conjunto de soluciones posibles asociado a $v$ es $SP_v$:
\begin{displaymath}
  SP_v = \left\{ { \sum_{u \in origenes} \left\vert{P_{u,v}}\right\vert : P_{u, v} \in PS_{u, v}} \right\}
\end{displaymath}
La solución óptima para un nodo $v$ la denominaremos $SP_v^{opt}$ y está determinada de la siguiente forma:
\begin{displaymath}
  SP_v^{opt} = \min \left\{ {s : s \in SP_v} \right\} 
\end{displaymath}

\begin{lema}
\label{lema_p2}
Una solución óptima para un nodo $v$ está compuesta por caminos mínimos entre los orígenes
y $v$:
\begin{displaymath}
  s = SP_v^{opt} \Leftrightarrow s = \sum_{u \in origenes} \min \left\{ { \left\vert{P}\right\vert : P \in PS_{u, v}} \right\}
\end{displaymath}
\end{lema}
\begin{proof}
  Sea $SP_{opt}$ una solución óptima para el nodo $v$ que contiene un camino $P_{w,v}$ no mínimo. Sea $P_{w,v}^{min}$ 
un camino mínimo entre $w \in origenes$ y $v$. Sea $SP'_{opt}$:
\begin{displaymath}
SP'_{opt} = SP_{opt} \setminus P_{w, v} \cup \left\{{P_{w,v}^{min}}\right\}
\end{displaymath}
veamos el costo de $SP'_{opt}$ expresado en función del de $SP_{opt}$:
\begin{displaymath}
  SP'_{opt} = \sum_{u \in origenes} \left\vert{P_{u,v}}\right\vert - \left\vert{P_{w,v}}\right\vert + \left\vert{P_{w,v}}\right\vert
\end{displaymath}
pero dado que $P_{w,v}$ no es mínimo:
\begin{displaymath}
  P_{w,v} > P_{w,v}^{min} \Rightarrow P_{w,v} - P_{w,v}^{min} > 0
\end{displaymath}
pero entonces volviendo a la ecuación anterior nos queda que:
\begin{displaymath}
  SP'_{opt} < SP_{opt}
\end{displaymath}
Lo cual es absurdo porque partimos suponiendo que $SP_{opt}$ era un solución óptima y por lo tanto mínima.
\end{proof}

Finalmente, caracterizaremos a la solución del problema $S$:
\begin{displaymath}
  S = \min_{v \in V} \left\{ {SP_{v}^{opt}} \right\}
\end{displaymath}
Nuestro algoritmo calcula el peso del camino mínimo entre cada uno de los orígenes y el resto de los nodos (TODO agregar cita
al cormen capítulo 22 página 598). 
Luego suma en cada nodo $v$ el peso de los caminos mínimos desde cada uno de los orígenes hasta él que por \textbf{Lema 1} es
igual a $SP_{v}^{opt}$. Por último, recorre todos los nodos y toma el mínimo entre todos los $SP_{v}^{opt}$ calculados
lo cual es precisamente la solución que acabamos de definir.

\subsection{Análisis de complejidad}
El algoritmo tiene 3 partes importantes a considerar con respecto a su complejidad:
\begin{enumerate}
  \item generar el grafo que modela la instancia del problema
  \item aplicar BFS una vez por cada caballo con el nodo correspondiente al caballo como origen
  \item buscar el nodo que cuente con la suma mínima
\end{enumerate}
Generar el grafo que modela la instancia del problema recibida tiene costo $O(\left\vert{V}\right\vert) = O(n^2)$. 
Sólo precisamos una matriz de tamaño $n \times n$. Cada posición representa a un nodo y en él almacenaremos la suma
de los caminos mínimos desde cada uno de los caballos hasta él. No es necesario guardar información acerca de las
aristas del grafo porque, al ser el salto de del caballo un movimiento conocido, dado un nodo se pueden obtener sus vecinos
en tiempo constante $O(1)$ calculándolos \textit{ad hoc}:
\begin{center}
(TODO insertar imagen de un tablero con una C en el centro, otra en un rincón y otra en una banda, indicando en cada
caso los saltos posibles)
\end{center}
Tampoco es necesario guardar la distancia desde cada origen hasta cada uno de los nodos ya que sólo nos interesará
la suma total sobre cada nodo. Por lo tanto, basta con ir sumando la distancia de cada uno de los orígenes hasta él.\\
El costo de aplicar BFS es $O(\left\vert{E}\right\vert) = O(n^2 * 8) = O(n^2)$ (TODO agregar cita al cormen) porque, 
como ya vimos, la cantidad de aristas del grafo está acotada por la cantidad de nodos, porque el caballo a lo sumo
puede saltar a 8 casillas diferentes.\\
Por último, buscar el mínimo tiene un costo $O(\left\vert{V}\right\vert) = O(n^2)$ porque simplemente consiste
en iterar por todos los nodos y quedarse con el que tenga la suma mínima.\\
Por lo tanto, la complejidad está dada por:
\begin{displaymath}
  O(construir\_grafo) + \#caballos * O(BFS) + O(obtener\_minimo) = 
\end{displaymath}
\begin{displaymath}
  O(n^2) + k * O(n^2) + O(n^2) = O(k * n^2)
\end{displaymath}

\subsection{Tests de complejidad}
