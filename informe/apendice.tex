\section{Apéndice: Códigos fuente}

\subsection{Problema 2: Plan de vuelo }
\lstset{language=Java,
                keywordstyle=\color{blue},
                stringstyle=\color{red},
                commentstyle=\color{magenta},
                morecomment=[l][\color{magenta}]{\#}
}
\begin{lstlisting}[frame=single]
public static String mejorVuelo(Aeropuerto[] aeropuertos) {
		if (existeVuelo(aeropuertos, aeropuertos[0], aeropuertos[1], -2)) {
			Vuelo min = aeropuertos[1].primeroEnLlegar();
			StringBuilder builder = new StringBuilder();
			builder.insert(0, " " + min.id());
			String llegada = min.llegada() + " ";
			int k = 1;
			while (!min.origen().equals(aeropuertos[0])) {
				min = min.origen().primeroEnLlegar();
				builder.insert(0, " " + min.id());
				k++;
			}
			return llegada + k + builder.toString();
		}
		return "no";
	}

public static boolean existeVuelo(Aeropuerto[] aeropuertos,
			Aeropuerto inicio, Aeropuerto destino, int t) {
		if (inicio.equals(destino)) {
			return true;
		}
		if (t + 2 <= inicio.obtenerUltimoVueloQueLlega()) {
			return true;
		}
		boolean llego = false;
		List<Vuelo> vuelos = inicio.vuelosQueSalen();
		List<Vuelo> vuelosNoAnalizados = new LinkedList<Vuelo>();
		inicio.vaciarVuelosQueSalen();
		for (Vuelo vuelo : vuelos) {
			if (vuelo.partida() >= t + 2) {
				if (vuelo.color().equals(Color.BLANCO)) {
					if (existeVuelo(aeropuertos, vuelo.destino(), destino,
							vuelo.llegada())) {
						vuelo.cambiarColor(Color.VERDE);
						if (vuelo.destino().primeroEnLlegar() == null
								|| vuelo.destino().primeroEnLlegar().llegada() > vuelo
										.llegada()) {
							vuelo.destino().agregarPrimeroEnLlegar(vuelo);
						}
						if (inicio.obtenerUltimoVueloQueLlega() < vuelo
								.llegada()) {
							inicio.cambiarUltimoVueloQueLlega(vuelo);
						}
						llego = true;
					} else {
						vuelo.cambiarColor(Color.ROJO);
					}
				}
			} else {
				vuelosNoAnalizados.add(vuelo);
			}
		}
		inicio.asignarVuelosQueLlegan(vuelosNoAnalizados);
		return llego;
	}

\end{lstlisting}
\subsection{Problema 3: La comunidad del anillo}
\subsubsection{Código del algoritmo que resuelve el problema}
\lstset{language=C++,
                keywordstyle=\color{blue},
                stringstyle=\color{red},
                commentstyle=\color{magenta},
                morecomment=[l][\color{magenta}]{\#}
}
\begin{lstlisting}[frame=single]
typedef int Vertice;

class Arista {
    private:
        Vertice v1;
        Vertice v2;
        int l;
    public:
        Arista() : v1(-1), v2(-1), l(-1){}
        Arista(const Vertice & v1, const Vertice & v2, const int & l) : v1(v1), v2(v2), l(l){}
        int costo() const {
            return l;
        }
        bool incideEn(Vertice v) const {
            return (v == v1 || v == v2) ? true : false;
        }
        pair<bool,Vertice> incideEnDosVertices(const set<Vertice> & conjVertices) const {
            pair<bool,Vertice> res;
            int incideEnV1 = conjVertices.count(v1);
            int incideEnV2 = conjVertices.count(v2);
            if (incideEnV1 + incideEnV2 == 2) {
                res.first = true;
            } else {
                res.first = false;
                res.second = (incideEnV1 == 0) ? v1 : v2;
            }
            return res;
        }
        pair<Vertice,Vertice> dameVertices() const {
            return pair<Vertice,Vertice>(v1,v2);
        }
        Vertice dameVerticeUno() const {
            return v1;
        }
        Vertice dameVerticeDos() const {
            return v2;
        }
        Vertice dameElOtroVertice(Vertice v) const { // Requiere que (v == v1) ó (v == v2)
            return (v == v1) ? v2 : v1;
        }
};

struct comparacionArista {
    bool operator() (const Arista & lhs, const Arista & rhs) const {
        // Quiero que:
        // Si tienen = costo, ordene por costo
        // Si tienen != costo, ordene por vértices (excepto el caso particular en que sean e = (a,b,c), e' = (b,a,c); entonces e = e')
        if (lhs.costo() == rhs.costo()) { // (x1,y1,c), (x2,y2,c)
            if( lhs.dameVerticeUno() == rhs.dameVerticeUno() ) { // (a,y1,c), (a,y2,c)
                return lhs.dameVerticeDos() < rhs.dameVerticeDos();
            } else { // (a,y1,c), (b,y2,c) con a != b
                if (lhs.dameVerticeUno() == rhs.dameVerticeDos() && lhs.dameVerticeDos() == rhs.dameVerticeUno()) {
                    return false; // (a,b,c) y (b,a,c) son la misma arista
                } else {
                    return lhs.dameVerticeUno() < rhs.dameVerticeUno();
                }
            }
        } else {
            return lhs.costo() < rhs.costo();
        }
    }
};

enum Color {BLANCO, GRIS, NEGRO};

struct infoVerticeDFS {
    Vertice verticeAnterior;
    Arista aristaAnterior;
    Color estado;
    list<Arista> backEdges;
    infoVerticeDFS() : verticeAnterior(-1), aristaAnterior(), estado(BLANCO), backEdges() {}
};

void DFS_visit(vector< list<Arista> > & aristasDeCadaVerticeAGM, infoVerticeDFS * info, Vertice actual) {
    info[actual].estado = GRIS;
    for (auto it = aristasDeCadaVerticeAGM[actual].begin(); it != aristasDeCadaVerticeAGM[actual].end(); it++) {
        Vertice nuevoActual = it->dameElOtroVertice(actual);
        if (info[nuevoActual].estado == BLANCO) {
            info[nuevoActual].aristaAnterior = *it;
            info[nuevoActual].verticeAnterior = actual;
            DFS_visit(aristasDeCadaVerticeAGM, info, nuevoActual);
        } else if (info[nuevoActual].estado == GRIS) {
            if (nuevoActual == info[actual].verticeAnterior) {
                /* Si estoy acá es porque estaba volviendo por la misma arista que vine (formando un circuito NO simple) */
            } else {
                /* El nuevo vertice ya habia sido visitado, entonces hay back edge */
                info[nuevoActual].backEdges.push_back(*it);
            }
        }
    }
    info[actual].estado = NEGRO;
}

int main(int argc, const char* argv[]) {
    unsigned int n, m, costoTotal = 0; // n = #vertices, m = #aristas
    vector< list<Arista> > aristasDeCadaVertice;
    vector< list<Arista> > aristasDeCadaVerticeAGM;
    set<Arista, comparacionArista> aristasGrafo;
    set<Vertice> verticesAGM;
    set<Arista, comparacionArista> aristasAGM;
    set<Arista, comparacionArista> aristasCandidatasAGM;
    list<Arista> aristasAnillo;
    
    cin >> n >> m;
    
    if (m < n) { // Si hay menos de n aristas, no existe solución (ver Correctitud)
        cout << "no" << endl;
        return 0;
    }
    
    for (unsigned int i = 0; i < n; i++) {
        aristasDeCadaVertice.push_back(list<Arista>());
        aristasDeCadaVerticeAGM.push_back(list<Arista>());
    }
    
    for (unsigned int i = 0; i < m; i++) {
        Vertice v1, v2;
        int l;
        cin >> v1 >> v2 >> l;
        v1--; v2--; // Como los equipos van de 1 a n, resto uno para que v1 y v2 vayan de 0 a n-1. Al devolver la solución sumo 1 y listo
        Arista a(v1, v2, l);
        aristasDeCadaVertice[v1].push_back(a);
        aristasDeCadaVertice[v2].push_back(a);
        aristasGrafo.insert(a);
    }
    
    // Arranco poniendo el vértice 0 en verticesAGM, y sus aristas en aristasCandidatasAGM
    verticesAGM.insert(0);
    for(auto it = aristasDeCadaVertice[0].begin(); it != aristasDeCadaVertice[0].end(); it++) {
        aristasCandidatasAGM.insert(*it);
    }
    
    while (aristasAGM.size() < n - 1 && aristasCandidatasAGM.size() > 0 ) {
        auto iterAristaMinima = aristasCandidatasAGM.begin();
        Arista a = *iterAristaMinima;
        aristasCandidatasAGM.erase(iterAristaMinima);
        pair<bool,Vertice> infoIncidencia = a.incideEnDosVertices(verticesAGM);
        if (infoIncidencia.first) { // Si incide en 2 vertices, no la puedo usar
            continue;
        } else {
            Vertice nuevo = infoIncidencia.second; // Este es el vértice en el que no incide, no está en AGM
            verticesAGM.insert(nuevo); 
            aristasGrafo.erase(a);
            aristasAGM.insert(a); 
            Vertice otro = a.dameElOtroVertice(nuevo);
            aristasDeCadaVerticeAGM[nuevo].push_back(a); 
            aristasDeCadaVerticeAGM[otro].push_back(a);
            for(auto it = aristasDeCadaVertice[nuevo].begin(); it != aristasDeCadaVertice[nuevo].end(); it++) {
                if (it->incideEnDosVertices(verticesAGM).first) {
                    continue; // Si estoy acá, iba a agregar una arista repetida o ya considerada
                }
                aristasCandidatasAGM.insert(*it);
            }
        }
    }
    
    if (aristasAGM.size() < n - 1) { // Si el árbol no tiene n - 1 aristas, no es generador (el grafo original no era conexo)
        cout << "no" << endl;
        return 0;
    }
    
    // Ahora agrego la arista con menor peso:
    if (aristasGrafo.size() == 0) { // En aristasGrafo quedaron las aristas que no puse en el AGM
        cout << "no" << endl;
        return 0;
    }
    Arista menor = *aristasGrafo.begin();
    aristasAGM.insert(menor);
    for (auto it = aristasAGM.begin(); it != aristasAGM.end(); it++) {
        costoTotal += it->costo();
    }
    Vertice primero = menor.dameVerticeUno();
    Vertice segundo = menor.dameVerticeDos();
    aristasDeCadaVerticeAGM[primero].push_back(menor);
    aristasDeCadaVerticeAGM[segundo].push_back(menor);
    // Tengo que encontrar el circuito simple, esto es, el anillo
    infoVerticeDFS info[n];
    DFS_visit(aristasDeCadaVerticeAGM, info, primero); // Llamo a DFS con un vértice que sé que está en el ciclo
    if (info[primero].backEdges.size() != 1) {
        cout << "Hay algo mal, el vertice deberia tener exactamente un back edge." << endl;
        return 1;
    }
    Arista backEdge = info[primero].backEdges.front();
    aristasAnillo.push_back(backEdge);
    aristasAGM.erase(backEdge);
    Vertice actual = backEdge.dameElOtroVertice(primero);
    while (actual != primero) {
        aristasAGM.erase(info[actual].aristaAnterior); // En aristasAGM van a quedar las aristas fuera del circuito
        aristasAnillo.push_back(info[actual].aristaAnterior); // Lo contrario para aristasAnillo
        actual = info[actual].verticeAnterior;
    }
    cout << costoTotal << " " << aristasAnillo.size() << " " << aristasAGM.size() << endl;
    for (auto it = aristasAnillo.begin(); it != aristasAnillo.end(); it++) {
        cout << it->dameVerticeUno() + 1 << " " << it->dameVerticeDos() + 1 << endl;
    }
    for (auto it = aristasAGM.begin(); it != aristasAGM.end(); it++) {
        cout << it->dameVerticeUno() + 1 << " " << it->dameVerticeDos() + 1 << endl;
    }
    
    return 0;
}
\end{lstlisting}
\subsubsection{Código del generador de grafos aleatorios}
\lstset{language=C++,
                keywordstyle=\color{blue},
                stringstyle=\color{red},
                commentstyle=\color{magenta},
                morecomment=[l][\color{magenta}]{\#}
}
\begin{lstlisting}[frame=single]
Vertice seleccionarVerticeRandom(set<Vertice> & conjunto) {
    int i = rand() % conjunto.size();
    auto it = conjunto.begin();
    for(int j = 0; j < i; j++) it++;
    return *it;
}

int main(int argc, const char* argv[]) {
    srand(time(NULL) + getpid()); // Seedeo
    for (int n = 1; n <= MAX_VERTICES; n++) {
        for (int i = 1; i <= CANT_INSTANCIAS; i++) {
            int m = rand() % (1 + n * (n - 1) / 2); // m es un valor aleatorio entre 0 y n(n-1)/2
            cout << n << " " << m << endl;
            set<Vertice> vertices; // Acá voy a tener el conjunto de vértices que todavía al menos una arista disponible.
            for (int i = 0; i < n; i++) {
                vertices.insert(i); // Los vertices van a ser enteros entre 0 y n-1, tengo que sumarles uno al imprimir
            }
            vector< set<Vertice> > vecinosPosibles(n);
            for (int i = 0; i < n; i++) {
                for (int j = 0; j < n; j++) {
                    if (i != j) {
                        vecinosPosibles[i].insert(j);
                    }
                }
            }
            while (m > 0) {
                Vertice v = seleccionarVerticeRandom(vertices);
                if (vecinosPosibles[v].size() == 0) {
                    vertices.erase(v);
                } else {
                    m--;
                    Vertice w = seleccionarVerticeRandom(vecinosPosibles[v]);
                    vecinosPosibles[v].erase(w);
                    vecinosPosibles[w].erase(v);
                    cout << v + 1 << " " << w + 1 << " " << (rand() % 100) + 1 << endl;
                }
            }
        }
    }
}
\end{lstlisting}
\subsubsection{Código del generador de grafos completos}
\lstset{language=C++,
                keywordstyle=\color{blue},
                stringstyle=\color{red},
                commentstyle=\color{magenta},
                morecomment=[l][\color{magenta}]{\#}
}
\begin{lstlisting}[frame=single]
int main(int argc, const char* argv[]) {
    srand(time(NULL) + getpid()); // Seedeo
    for (int n = 1; n <= MAX_VERTICES; n++) {
        for (int i = 1; i <= CANT_INSTANCIAS; i++) {
            int m = n * (n - 1) / 2;
            cout << n << " " << m << endl;
            for (int v = 1; v <= n; v++) {
                for (int a = 1; a <= (n-1) - (v-1); a++) {
                    cout << v << " " << (v + a) << " " << (rand() % 100) + 1 << endl;
                }
            }
        }
    }
}
\end{lstlisting}
\subsubsection{Código del tester (sólo lo relevante, el resto es igual)}
\lstset{language=C++,
                keywordstyle=\color{blue},
                stringstyle=\color{red},
                commentstyle=\color{magenta},
                morecomment=[l][\color{magenta}]{\#}
}
\begin{lstlisting}[frame=single]
chrono::microseconds algoritmo(unsigned int n, unsigned int m, unsigned int costoTotal, vector< list<Arista> > aristasDeCadaVertice, vector< list<Arista> > aristasDeCadaVerticeAGM, set<Arista, comparacionArista> aristasGrafo, set<Vertice> verticesAGM, set<Arista, comparacionArista> aristasAGM, set<Arista, comparacionArista> aristasCandidatasAGM, list<Arista> aristasAnillo) 
{
    auto start_time = chrono::high_resolution_clock::now();
    // Arranco poniendo el vértice 0 en verticesAGM, y sus aristas en aristasCandidatasAGM
    verticesAGM.insert(0);
    for(auto it = aristasDeCadaVertice[0].begin(); it != aristasDeCadaVertice[0].end(); it++) {
        aristasCandidatasAGM.insert(*it);
    }
    
    while (aristasAGM.size() < n - 1 && aristasCandidatasAGM.size() > 0 ) {
        auto iterAristaMinima = aristasCandidatasAGM.begin();
        Arista a = *iterAristaMinima;
        aristasCandidatasAGM.erase(iterAristaMinima);           
        pair<bool,Vertice> infoIncidencia = a.incideEnDosVertices(verticesAGM);
        if (infoIncidencia.first) {                             
            continue;
        } else {
            Vertice nuevo = infoIncidencia.second;              
            verticesAGM.insert(nuevo);                          
            aristasGrafo.erase(a);                              
            aristasAGM.insert(a);                               
            Vertice otro = a.dameElOtroVertice(nuevo);
            aristasDeCadaVerticeAGM[nuevo].push_back(a);        
            aristasDeCadaVerticeAGM[otro].push_back(a);
            for(auto it = aristasDeCadaVertice[nuevo].begin(); it != aristasDeCadaVertice[nuevo].end(); it++) {
                if (it->incideEnDosVertices(verticesAGM).first) {
                    continue;                                   
                }
                aristasCandidatasAGM.insert(*it);
            }
        }
    }
    
    if ( (aristasAGM.size() < n - 1) || (aristasGrafo.size() == 0)) {
        // No hay solucion
    } else {
        Arista menor = *aristasGrafo.begin();
        aristasAGM.insert(menor);
        for (auto it = aristasAGM.begin(); it != aristasAGM.end(); it++) {
            costoTotal += it->costo();
        }
        Vertice primero = menor.dameVerticeUno();
        Vertice segundo = menor.dameVerticeDos();
        aristasDeCadaVerticeAGM[primero].push_back(menor);
        aristasDeCadaVerticeAGM[segundo].push_back(menor);
        // Tengo que encontrar el circuito simple, esto es, el anillo
        infoVerticeDFS info[n];
        DFS_visit(aristasDeCadaVerticeAGM, info, primero);
        if (info[primero].backEdges.size() != 1) {
            cout << "Hay algo mal, el vertice deberia tener exactamente un back edge." << endl;
        }
        Arista backEdge = info[primero].backEdges.front();
        aristasAnillo.push_back(backEdge);
        aristasAGM.erase(backEdge);
        Vertice actual = backEdge.dameElOtroVertice(primero);
        while (actual != primero) {
            aristasAGM.erase(info[actual].aristaAnterior);
            aristasAnillo.push_back(info[actual].aristaAnterior);
            actual = info[actual].verticeAnterior;
        }
    }
    
    auto end_time = chrono::high_resolution_clock::now();
    chrono::microseconds tiempo = chrono::duration_cast<chrono::microseconds>(end_time - start_time);
    return tiempo;
}

const int CANT_INSTANCIAS = 1000;
const int MAX_VERTICES = 100;
const int CANT_REPETICIONES_CALCULO_INSTANCIA = 10;

int main(int argc, const char* argv[]) {
    for (int cantVertices = 1; cantVertices <= MAX_VERTICES; cantVertices++) {
        int promedio = 0;
        for (int instancia = 1; instancia <= CANT_INSTANCIAS; instancia++) {
            unsigned int n, m, costoTotal = 0;
            vector< list<Arista> > aristasDeCadaVertice;
            vector< list<Arista> > aristasDeCadaVerticeAGM;
            set<Arista, comparacionArista> aristasGrafo;
            set<Vertice> verticesAGM;
            set<Arista, comparacionArista> aristasAGM;
            set<Arista, comparacionArista> aristasCandidatasAGM;
            list<Arista> aristasAnillo;
            
            cin >> n >> m;
            
            for (unsigned int i = 0; i < n; i++) {
                aristasDeCadaVertice.push_back(list<Arista>());
                aristasDeCadaVerticeAGM.push_back(list<Arista>());
            }
            
            for (unsigned int i = 0; i < m; i++) {
                Vertice v1, v2;
                int l;
                cin >> v1 >> v2 >> l;
                v1--; v2--;
                Arista a(v1, v2, l);
                aristasDeCadaVertice[v1].push_back(a);
                aristasDeCadaVertice[v2].push_back(a);
                aristasGrafo.insert(a);
            }
            
            chrono::microseconds minTiempo(INT_MAX);
            for (int rep = 1; rep <= CANT_REPETICIONES_CALCULO_INSTANCIA; rep++) {
                // Notar que todo se pasa por copia para que no se modifique.
                chrono::microseconds tiempoRep = algoritmo(n, m, costoTotal, aristasDeCadaVertice, aristasDeCadaVerticeAGM, aristasGrafo, verticesAGM,aristasAGM, aristasCandidatasAGM, aristasAnillo);
                if (tiempoRep < minTiempo)
                    minTiempo = tiempoRep;
            }
            
            promedio += minTiempo.count();
        }
        promedio = promedio / CANT_INSTANCIAS;
        cout << cantVertices << " " << promedio << endl;
    }
    
    
    return 0;
}
\end{lstlisting}
